%% Commands for TeXCount
%TC:macro \cite [option:text,text]
%TC:macro \citep [option:text,text]
%TC:macro \citet [option:text,text]
%TC:envir table 0 1
%TC:envir table* 0 1
%TC:envir tabular [ignore] word
%TC:envir displaymath 0 word
%TC:envir math 0 word
%TC:envir comment 0 0

\documentclass[acmsmall,screen,review]{acmart}

\usepackage{syntax}
\renewcommand{\syntleft}{\normalfont\itshape}
\renewcommand{\syntright}{\normalfont\itshape}

\usepackage{prftree}

\usepackage{listings}
\usepackage{xcolor}
\usepackage{subcaption}
\usepackage{fancyvrb}

\definecolor{codegreen}{rgb}{0,0.6,0}
\definecolor{codegray}{rgb}{0.5,0.5,0.5}
\definecolor{codepurple}{rgb}{0.58,0,0.82}
\definecolor{backcolour}{rgb}{0.95,0.95,0.92}

\lstdefinestyle{mystyle}{
%    backgroundcolor=\color{backcolour},   
    commentstyle=\color{codegreen},
    keywordstyle=\color{magenta},
    numberstyle=\tiny\color{codegray},
    stringstyle=\color{codepurple},
    basicstyle=\ttfamily\footnotesize,
    breakatwhitespace=false,         
    breaklines=true,                 
    captionpos=b,                    
    keepspaces=true,                 
    numbers=left,                    
    numbersep=5pt,                  
    showspaces=false,                
    showstringspaces=false,
    showtabs=false,                  
    tabsize=2
}

\lstset{style=mystyle}

\newcounter{todos}
\newcommand{\TODO}[1]{{
  \stepcounter{todos}
  \begin{center}\large{\textcolor{red}{\textbf{TODO \arabic{todos}:} #1}}\end{center}
}}
\newcommand{\sorry}{\textcolor{red}{\textbf{sorry}}}

\newcommand{\todo}[1]{\stepcounter{todos} \textcolor{red}{TODO \arabic{todos}: #1}}

% Math fonts
\newcommand{\mc}[1]{\ensuremath{\mathcal{#1}}}
\newcommand{\mb}[1]{\ensuremath{\mathbf{#1}}}
\newcommand{\ms}[1]{\ensuremath{\mathsf{#1}}}

% Math
\newcommand{\nats}{\mathbb{N}}

% Syntax atoms
\newcommand{\lbl}[1]{{`#1}}
\newcommand{\lto}{\Rightarrow}
\newcommand{\linl}[1]{\ms{inl}\;{#1}}
\newcommand{\linr}[1]{\ms{inr}\;{#1}}
\newcommand{\labort}[1]{\ms{abort}\;{#1}}

% Syntax
\newcommand{\letexpr}[3]{\ensuremath{\ms{let}\;#1 = #2\;\ms{in}\;#3}}
\newcommand{\letstmt}[3]{\ensuremath{\ms{let}\;#1 = #2; #3}}
\newcommand{\brb}[2]{\ms{br}\;#1\;#2}
\newcommand{\lbrb}[2]{\brb{\lbl{#1}}{#2}}
\newcommand{\ite}[3]{\ms{if}\;#1\;\{#2\}\;\ms{else}\;\{#3\}}
\newcommand{\case}[5]{\ms{case}\;#1\;\{\linl{#2} \lto #3, \linr{#4} \lto #5\}}
\newcommand{\where}[2]{#1\;\ms{where}\;\{#2\}}
\newcommand{\wbranch}[3]{#1(#2) \lto #3}
\newcommand{\lwbranch}[3]{\wbranch{\lbl{#1}}{#2}{#3}}
\newcommand{\bsplice}[3]{#1(#2)\;\{#3\}}
%\newcommand{\lbsplice}[3]{\bsplice{\lbl{#1}}{#2}{#3}}
\newcommand{\csplits}[3]{#1 \mapsto #2;#3}
\newcommand{\cwk}[2]{#1 \mapsto #2}
\newcommand{\lwk}[2]{#1 \rightsquigarrow #2}
\newcommand{\tlin}[2]{#2 \subseteq \ms{lin}(#1)}
\newcommand{\ltlin}[3]{#3 \subseteq \ms{lin}(#1) \cap #2}
\newcommand{\thyp}[3]{#1: {#2}^{#3}}
\newcommand{\lhyp}[3]{#1[#2](#3)}
\newcommand{\llhyp}[3]{\lhyp{\lbl{#1}}{#2}{#3}}
\newcommand{\rle}[1]{{\scriptsize\textsf{#1}}}
\newcommand{\taff}{{\{\ms{a}\}}}
\newcommand{\trel}{{\{\ms{r}\}}}
\newcommand{\tint}{{\{\ms{a}, \ms{r}\}}}
\newcommand{\hasty}[5]{#1 \vdash_{#2} #3: {#4}^{#5}}
\newcommand{\haslb}[3]{#1 \vdash #2 \rhd #3}
\newcommand{\lhaslb}[3]{#1 \vdash #2 \rhd #3}
\newcommand{\issubst}[3]{#1: #2 \mapsto #3}
\newcommand{\lbsubst}[3]{#1: #2 \rightsquigarrow #3}
\newcommand{\exprletsubst}[2]{{#1};{#2}}
\newcommand{\stmtletsubst}[2]{{#1};{#2}}
\newcommand{\mhole}[1]{{#1}^?}
\newcommand{\lhole}[1]{?#1}
\newcommand{\mhasty}[6]{#1;#2 \vdash_{#3} #4: {#5}^{#6}}
\newcommand{\mhaslb}[4]{#1;#2 \vdash #3 \rhd #4}
\newcommand{\mlhaslb}[4]{#1;#2 \vdash #3 \rhd #4}
\newcommand{\tyhole}[5]{#1: #2 \mapsto_{#3} {#4}^{#5}}
\newcommand{\blkhole}[3]{#1: #2 \mapsto #3}
\newcommand{\cfghole}[3]{#1: #2 \mapsto #3}
\newcommand{\substctx}[2]{{#1}^{#2}}
\newcommand{\substlbs}[2]{{#1}^{#2}}
\newcommand{\restrictsubst}[2]{{#1}_{#2}}
\newcommand{\subsubst}[2]{#1 \subseteq #2}
\newcommand{\isrw}[3]{#1: #2 \mapsto #3}
\newcommand{\mbind}{\mathbin{{>}\hspace{-0.1em}{>}\hspace{-0.1em}{=}}}
% \newcommand{\strictlbsubst}[3]{#1: #2 \rightsquigarrow_= #3}

% Denotational semantics
\newcommand{\dnt}[1]{\llbracket{#1}\rrbracket}
\newcommand{\ednt}[1]{\left\llbracket{#1}\right\rrbracket}
\newcommand{\upg}[2]{{#1}^{\uparrow #2}}

% Weak memory
\newcommand{\bufloc}[1]{\overline{#1}}

% Branding
\newcommand{\isotopessa}{\ms{isotope_{SSA}}}

%% Rights management information.  This information is sent to you
%% when you complete the rights form.  These commands have SAMPLE
%% values in them; it is your responsibility as an author to replace
%% the commands and values with those provided to you when you
%% complete the rights form.
\setcopyright{acmcopyright}
\copyrightyear{2024}
\acmYear{2024}
\acmDOI{XXXXXXX.XXXXXXX}

%%
%% These commands are for a JOURNAL article.
% \acmJournal{JACM}
% \acmVolume{37}
% \acmNumber{4}
% \acmArticle{111}
% \acmMonth{8}

%%
%% Submission ID.
%% Use this when submitting an article to a sponsored event. You'll
%% receive a unique submission ID from the organizers
%% of the event, and this ID should be used as the parameter to this command.
%%\acmSubmissionID{123-A56-BU3}

\begin{document}

\title{The Denotational Semantics of SSA}

\author{Jad Ghalayini}
\email{jeg74@cl.cam.ac.uk}
\orcid{0000-0002-6905-1303}

\author{Neel Krishnaswami}
\email{nk480@cl.cam.ac.uk}
\orcid{0000-0003-2838-5865}

\begin{abstract}
  ...
\end{abstract}

\begin{CCSXML}
  <ccs2012>
  <concept>
  <concept_id>10003752.10010124.10010131.10010133</concept_id>
  <concept_desc>Theory of computation~Denotational semantics</concept_desc>
  <concept_significance>500</concept_significance>
  </concept>
  <concept>
  <concept_id>10003752.10010124.10010131.10010137</concept_id>
  <concept_desc>Theory of computation~Categorical semantics</concept_desc>
  <concept_significance>500</concept_significance>
  </concept>
  <concept>
  <concept_id>10003752.10003790.10011740</concept_id>
  <concept_desc>Theory of computation~Type theory</concept_desc>
  <concept_significance>500</concept_significance>
  </concept>
  <concept>
  <concept_id>10003752.10003790.10011742</concept_id>
  <concept_desc>Theory of computation~Separation logic</concept_desc>
  <concept_significance>300</concept_significance>
  </concept>
  </ccs2012>
\end{CCSXML}

\ccsdesc[500]{Theory of computation~Denotational semantics}
\ccsdesc[500]{Theory of computation~Categorical semantics}
\ccsdesc[500]{Theory of computation~Type theory}
\ccsdesc[300]{Theory of computation~Separation logic}

%%
%% Keywords. The author(s) should pick words that accurately describe
%% the work being presented. Separate the keywords with commas.
\keywords{SSA, Categorical Semantics, Elgot Structure, Effectful Category}

% \received{20 February 2007}
% \received[revised]{12 March 2009}
% \received[accepted]{5 June 2009}

\maketitle

\section{Introduction}

Static single assignment form, or SSA form, has been the dominant
compiler intermediate representation since its introduction by
\citet{...} in the later 1980s. Every major compiler -- GCC, Clang,
MLIR, Cranelift -- uses this representation, because it makes many
optimizations much easier to do than traditional 3-address code IRs.

The key idea behind SSA is to adapt an idea from functional
programming: namely, every variable is defined only once. This means
that substitution is unconditionally valid, without first requiring a
dataflow analysis to compute where definitions reach. Unlike in
functional programming, though, scoping of definitions in SSA is
traditionally not lexical. Instead, scoping is determined by
\emph{dominance}: every variable occurence must be dominated by a
single assignment in the control flow graph.

The semantics of SSA has traditionally been handled quite informally,
because conceptually, it is a simple first-order imperative
programming language. As a result, whether a rewrite is sound or
not is usually obvious, without having to do a complex correctness
argument.

Unfortunately, computers are no longer as simple as they were in the
late 1980s. Modern computers are typically multicore, and feature many
levels of caching, and as a result the semantics of memory is no
longer correctly modelled as a big array of bytes. Finding good
semantics for modern weak memory systems remains an ongoing challenge.

As a result, it is not correct to justify compiler optimizations in
terms of a simple imperative model, and it is an open question which
equations should hold of an SSA program. This is a particularly
fraught question, because it is also unclear which equations weak
memory models should satisfy.

What we would like to know which equations any SSA representation
should satisfy. This would let us establish a contract between
compiler writers and hardware designers. The compiler writers could
rely upon the equational theory of SSA when justifying optimizations,
without needing to know all the details of the memory model at all
times.  Conversely, memory models could be validated by seeing if they
satisfy the equations of SSA, without needing to study every possible
compiler optimization.

Concretely, our contributions are as follows: 

\begin{itemize}
\item First, we give a type-theoretic presentation of SSA, with both typing
  rules and an equational theory for well-typed terms. We also prove the
  correctness of suitable substitution properties for this calculus. 
  
\item Next, we give a categorical semantics for this type theory, in
  terms of distributive Elgot categories. We show that any
  denotational model with this categorical structure is also a model
  of SSA. This shows that all of the equations we give are sound with
  respect to the categorical structure. 

\item We also show that syntax quotiented by the equational theory
  yields the initial distributive Elgot category. This establishes
  that our set of syntactic equations is complete, and that there are
  no equations which the denotational semantics validates, but which
  cannot be proved syntactically. 

\item We show that this denotational axiomatization is useful in
  practice, by giving a variety of concrete models, including a model
  of TSO weak memory based on~\cite{sparky}. This demonstrates that it
  is possible to give realistic weak memory models which do not
  disturb the structure of SSA in fundamental ways.

\item Finally, we have substantially mechanized our proofs. We have
  mechanized proofs of substitution for our type theory, as well as
  proofs that the syntax forms the initial model, and that the SPARC TSO
  semantics forms a valid model of SSA. The denotational semantics and
  its proof of the soundness of substitution are done on paper. 
\end{itemize}

\bibliographystyle{ACM-Reference-Format}
\bibliography{references}

\clearpage 

\appendix

\begin{figure}
  \begin{center}
    \begin{grammar}
      <\(A, B, C\)> ::= 
      \(X\)
      \;|\; \(A \otimes B\)
      \;|\; \(\mathbf{1}\)
      \;|\; \(A + B\)
      \;|\; \(\mathbf{0}\)

      <\(a, b, c, e\)> ::= \(x\) 
      \;|\; \(f\;a\)
      \;|\; \((a, b)\) 
      \;|\; \(()\) 
      \;|\; \(\linl{a}\) 
      \;|\; \(\linr{a}\)
      \;|\; \(\labort{a}\)
      % \;|\; \(\letexpr{x}{a}{e}\)
      % \;|\; \(\letexpr{(x, y)}{a}{e}\)
      % \;|\; \(\lbsplice{\ell}{x: A}{t}\)
      
      <\(s, t\)> ::= \(\lbrb{\ell}{a}\) 
      \;|\; \(\case{e}{x}{s}{y}{t}\)
      \;|\; \(\letstmt{x}{a}{t}\)
      \;|\; \(\letstmt{(x, y)}{a}{t}\)
      \;|\; \(\where{t}{L}\)

      <\(L\)> ::= \(\cdot\) \;|\; \(\lwbranch{\ell}{x: A}{t}, L\)

      <\(\Gamma\)> ::= \(\cdot\) \;|\; \(\Gamma, \thyp{x}{A}{\epsilon}\)

      <\(\ms{L}\)> ::= \(\cdot\) \;|\; \(\ms{L}, \lbl{\ell}(A)\)
    \end{grammar}
  \end{center}
  \caption{Grammar for \isotopessa, parametrized over a set of instructions \(f \in \mc{I}\)}
  \Description{Grammar for isotope-SSA}
  \label{fig:ssa-grammar}
\end{figure}

\end{document}
\endinput
